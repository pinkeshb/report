\renewcommand{\abstractname}{Abstract}
\begin{abstract}
\doublespacing
One important goal of image synthesis research is to accelerate the
process of obtaining realistic images using the radiosity method.
Two important concepts recently introduced are the general frame-
work of projection methods and the hierarchical radiosity method.
Wavelet theory, which explores the space of hierarchical basis
functions, offers an elegant framework that unites these two con-
cepts and allows us to more formally understand the hierarchical
radiosity method.
Wavelet expansions of the radiosity kernel have negligible en-
tries in regions where high frequency/fine detail information is
not needed. A sparse system remains if these entries are ignored.
This is similar to applying a lossy compression scheme to the form
factor matrix. The sparseness of the system allows for asymptoti-
cally faster radiosity algorithms by limiting the number of matrix
terms that need to be computed. The application of these methods
to 3D environments is described in [9]. Due to space limitations
in that paper many of the subtleties of the construction could not
be explored there. In this paper we discuss some of the mathemat-
ical details of wavelet projections and investigate the application
of these methods to the radiosity kernel of a flatland environment,
where many aspect are easier to visualize. 
%\\ \\
%\noindent \textbf{Keywords:-} Filter banks (FB), Wavelets, Orthogonal filter banks, Semidefinite programming (SDP), Finite impulse response (FIR), Vanishing moments (VM), Double shift orthogonality (DSO), Perfect reconstruction (PR), Time-frequency localization.
\end{abstract}  