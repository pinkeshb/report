
\subsection{\label{sub:EigenfilterApproach}The Eigenfilter Approach}
The eigenfilter approach \cite{key-43,key-14} is an efficient and popular method to design
digital filters. The technique gives an optimal solution as filter coefficients, which minimizes a quadratic cost function. The approach employs the computation of the filter as the eigenvector of an appropriate Hermitian positive-definite matrix associated with the cost function. Slepian \cite{key-43} introduced the notion which formed a precursor
to eigenfilter design. He designed a real window with minimum stop-band energy called a prolate spheroidal wave sequence. In designing
the prolate spheroidal wave sequence, the objective function $\phi$,
which is the energy of the sequence in the frequency range $\sigma\leq\omega\leq\pi$
for $0<\sigma<\pi$ is expressed in quadratic form and the optimization
problem is formalized as:
\begin{equation*}
\begin{aligned}
& \underset{\mathbf{a}} {\text{minimize}}
& & \phi=\mathbf{a}^{T}\mathbf{Pa} \\
& \text{subject to}
& & \mathbf{a}^{T}\mathbf{a}=1 \\
\end{aligned}
\end{equation*}
where $\mathbf{P}$ is a real, symmetric, Toeplitz and positive-definite
matrix and $\mathbf{a}$ is a real valued eigenvector. The optimal sequence,
which minimizes the cost function, is the eigenvector $\mathbf{a}$ of
the matrix $\mathbf{P}$ corresponding to its smallest eigenvalue.
The constraint $\mathbf{a}^{T}\mathbf{a}=1$, is imposed to avoid
trivial solutions.

Vaidyanathan and Nguyen \cite{key-14} presented  the {}``eigenfilter''
method to design an optimal FIR filter directly. In the eigenfilter approach, the filter is obtained
as the eigenvector of a real, symmetric, positive-definite matrix,
in a similar way to the prolate spheroidal window sequence design problem.
In the eigenfilter approach the objective function to be minimized
is the error between desired frequency response and the frequency
response of the filter to be designed. Eigenfilter-based
methods have also been used to design filter banks by a few authors \cite{key-15,key-26,key-27,key-29}.
\subsubsection{\label{sub:RayleighPrinciple}The Rayleigh's Principle \cite{key-13}}
For a function $\phi(\mathbf{x}) = \mathbf{x}^{T}\mathbf{Qx}$,
where $\mathbf{Q}$ is a real, symmetric, Toeplitz and positive-definite
$n$ by $n$ matrix and $\mathbf{x}$ is a real valued vector.\\

Then
\begin{equation*}
\begin{aligned}
& \underset{\mathbf{x}} {\text{min}}& \phi(\mathbf{x})= \underset{\mathbf{{x}^{T}x=1}} {\text{min}} & \phi(\mathbf{x})
\end{aligned}
\end{equation*}

According to Rayleigh's Principle
\begin{equation*}
\begin{aligned}
\underset{\mathbf{{x}^{T}\mathbf{x}=1}} {\text{min}} & \phi(\mathbf{x} ) = \lambda_1
\end{aligned}
\end{equation*}

where $$ \lambda_1\leq\lambda_2\leq\lambda_3........\leq\lambda_n$$ are the eigen values of matrix  $\mathbf{Q}$

The vector $\mathbf{x}$ which minimizes $\phi$ is the eigenvector corresponding to a minimum eigen value.
% 