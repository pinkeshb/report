\renewcommand{\abstractname}{Abstract}
% \vspace{-10cm}

\begin{abstract}
\doublespacing
Radiosity synthesize realistic images of a scene given geometric and optical properties of the scene with Lambertian surfaces. Implementation, testing and analysis of wavelet radiosity algorithm is performed.

Projection methods are used to approximately solve integral equations (IE) in finite  dimensional function space by casting problem to system of linear equations. Radiosity problem was solved by  casting it to radiosity integral equation, an inhomogeneous Fredholm equation of the second kind. Accuracy and time complexity of projection methods depend on the chosen finite dimensional function space and its basis. 
Different function spaces and basis has been tested with with different 2D, 3D scenes and integral equations with analytical solution. Space of piecewise polynomial functions, of order 
$m=0,1,2$ , over standard interval of chosen fixed size was chosen as function space.
 % Along with order of polynomials, m, different interval size was chosen for testing.
 Higher m and lower interval size results in higher accuracy.

Standard basis, shifted Legendre polynomials, is compared with wavelet basis, Haar wavelet, linear and quadratic Legendre multi-wavelets (in order of increasing moments of vanishing). Higher vanishing moments of wavelet results in larger number of negligible projection coefficients. Thus negligible coefficients were set to zero resulting in sparse system of linear equations that are solved faster at cost of increased error in solution. Trade-off between error and sparsity is analyzed for different basis. Higher vanishing moments results in higher sparsity, but at higher cost of  projection using quadrature rules.  

% Wavelet basis provides sparse system of linear equations with lower projection error, because of vanishing moments of wavelets. Algorithm is first tested with the IE with an analytical solution. Then different basis and spaces are compared for the accuracy of projection (after removing negligible coefficients) of kernel with 2D scenes knows as flatland scenes, where problem and 2D kernel is easier to visualize. Then the images for 3D scenes has been generated by projecting 4D kernel and solving resulting system of linear equations. 




%\\ \\
%\noindent \textbf{Keywords:-} Filter banks (FB), Wavelets, Orthogonal filter banks, Semidefinite programming (SDP), Finite impulse response (FIR), Vanishing moments (VM), Double shift orthogonality (DSO), Perfect reconstruction (PR), Time-frequency localization.
\end{abstract}  