\chapter{conclusion and future work}
In present work, projection methods, which are used to solve integral equations approximately, is used to solve radiosity integral equation for approximate solution, to synthesize realistic images of input scene. We have seen that the accuracy of projection methods highly depend on the chosen function space of approximate solution. While choice of basis for that function space changes the speed of algorithm by having sparse system of linear equations to solve. Different function spaces and basis has been tested for accuracy of projection and solution. function Space of piecewise polynomial function over standard interval of fixed size are used as function space. Along with different interval size $\{\frac{1}{4},\frac{1}{8},\frac{1}{16},\frac{1}{32},\frac{1}{64},\frac{1}{128},\frac{1}{256}\}$, different order of polynomials \{0,1,2\} were used for approximation. Higher the order of polynomials, higher the accuracy of approximation and lower the interval size, higher the accuracy. Standard basis of selected function space is compared with alternative basis i.e. wavelet basis. Haar wavelet, linear Legendre multi-wavelets and quadratic Legendre multi-wavelets are used, other wavelets like Chebyshev multi-wavelets CAS (cosine and sine wavelets) can be explored. An expansion of kernel of IE after projection using wavelet basis, leaves many coefficients negligible. We used this opportunity and set  these negligible coefficients to zero to get sparse system of linear equations. Thus speed of algorithm used for solving system of equations is increased. We used Jacobi iterative method to find inverse of the matrix of system of linear equations. We observed that speed comes with more error in projection and consequently error in solution increases. Increases in amount of error, due to removal of negligible coefficients, is analyzed for different chosen function space and standard and wavelet basis for chosen function space. we observed that wavelet basis provided sparse system of linear equations with lower projection error, because of vanishing moments of wavelets. Algorithm was successfully tested with the IE with an analytical solution. Then different basis (standard basis and wavelet basis) and function spaces are compared for the accuracy of projection (after removing negligible coefficients) of kernel with 2D scenes knows as flatland scenes, where problem and 2D kernel is easier to visualize. Chosen test scenes did not have any occlusion (which are normally present in scene). Thus the algorithm needs to be tested for scenes having occlusion in them. Then the images for 3D scenes has been generated by projecting 4D kernel and solving resulting system of linear equations. We observed that with increase in order of polynomials of basis functions we get highly accurate results with less number of basis functions. But with increase in order of polynomials results in increase in  cost of calculating projection coefficients. We used Gauss-Legendre quadrature rule which requires exponentially increasing number of sample points of 4 dimensional kernel, as order of polynomials is increased. Thus there is a trade-off between increasing order of polynomials or decreasing size of standard interval. Future work involves using more scenes to verify these results.  
