

\chapter{\label{ch:projection}Projection methods for radiosity problem}
In chapter \ref{ch:problemformulatio} we saw that problem of radiosity was casted to integral equation but it is very difficult to solve the integral equation analytically due to geometry of scene. We can use projection methods to find approximate solution. We first start with projecting the unknown radiosity function and integrate(operate kernel) on the projected function. After that we cast radiosity problem with system of linear equations. \\

\section{Projecting Radiosity Function}
We are working with linear operator (integral operator) defined on space of unknown function $B(x)$ we must first consider bases for this spaces. In general we will be dealing with functions of finite energy, i.e. functions
in the linear vector space $L^2$ . There are many possible bases for
this space. Consider some primal basis $\{N_i \}_{i{\bf belongs to}Z}$ . This set must contain infinitely many functions since $L^2$ is infinite dimensional.
By definition, every function $B{\bf belongs to}L^2$ can be written as a lin-
ear combination of the basis functions $B(x) = \sum _i b_i N_i(x)$ for
some coefficients $b_i$ . In order to find the coefficients of a given
function we need an inner product on $L^2$ . The inner product of
two functions F and G is defined as $<F, G> = \int F(x) G(y) ds$.
Throughout we will denote functions by capital letters. Their expansion coefficients with respect to a basis will be small letters
with indices as subscripts. When denoting functions we will often
drop the argument x (or y).
Given a function B and a set of basis functions $N_i$ , one way
of finding the coefficients of B with respect to this basis is by
performing inner products against the dual basis functions $N{\bf tilde}_i$
dual functions form a basis for the same space, and are defined
through the property\\
$<N_i,N{\bf tilde}_j>=\delta_{ij}$
where $\delta_{ij}$ is the Kronecker Delta. Using the duals we can write
the expansion of a function with respect to a basis as\\

$B(x) = \sum_ib_iN_i(x)= \sum_i<B,N_i>N_i(x)$\\

Now consider a linear operator defined on L 2 . One example is
integration against a kernel function, which is a linear operator since integration is linear. In particular we consider the radiosity
integral equation, which can be written as a linear operator\\

$B=E+KB$\\

where $KB(x) = \int K(x, y)B(y)dy$. Since K is a linear operator it
is fully described by its action on our chosen basis. Intuitively we
can write this as an infinite sized matrix. The columns of the matrix K are given by $KN_j$ . The coefficients of these columns with
respect to our basis can be found by taking inner products with
the dual functions. Thus the entries in the matrix representation
of K are given by\\

$K_{i,j}=<KN_j,N{\bf tilde}_i>=\int \int K(x,y)N_j(y)N{\bf tilde}_i(x)$\\

with this we can write radiosity integral equation as an infinite sized matrix equation\\

$(I-K)B=E$\\

{\bf Show countably finite thing \\ and also show the difference between space and basis for that space}\\

in next section we introduce wavelets and the wavelet basis
\\\\