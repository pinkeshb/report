

\chapter{\label{ch:projection}Projection Method for Radiosity Problem}
In chapter \ref{ch:problemformulatio} we saw that problem of radiosity was casted to integral equation but it is very difficult to solve the integral equation analytically due to geometry of scene. We can use projection methods to find approximate solution, Numerically. We first start with projecting the unknown radiosity function into finite dimension function space and integrate (operate kernel) on the projected radiosity function. After that we cast radiosity problem to system of linear equations which can be solved using computer. \\

\section{Projecting Radiosity Function}
We are working with linear operator (integral operator) defined on space of unknown function $B(x)$.  We first consider basis for this space. Since scene is finite so as the domain of surfaces, we are dealing with functions of finite energy, i.e. functions
in the linear vector space $L^2$ . There are many basis for
this space. Consider some general basis $\{N_i \}_{i \in \mathbb{N}}$ for $L^2$ space. Set contains infinitely many basis functions because $L^2$ is infinite dimensional space.
By definition, every function $B \in L^2$ can be written as a linear combination  $B(x) = \sum _i b_i N_i(x)$ of the basis functions for
some coefficients $b_i$ . In order to find these coefficients for projection of a given
function we define inner product on $L^2$ space. The inner product of
two functions F and G is defined as $\langle F, G \rangle  = \int F(x) G(y) dx$.
Given a function $B(x)$ and a set of orthogonal basis functions function $N_i(x)$ , we can find the coefficients of $B(x)$ with respect to this basis is by
calculating inner products with the dual basis functions $\tilde{N}_i(x)$. Dual functions can be used as basis for the same space, and are defined
through the property
\begin{equation}
\langle N_i(x),\tilde{N}_j(x) \rangle =\delta_{ij}
\end{equation}
where $\delta_{ij}$ is the Kronecker Delta. Using the dual basis we can expand a function  $B(x)$ with respect to a original basis as
\begin{equation}\label{eq:radproj}
B(x) = \sum_ib_iN_i(x)= \sum_i\langle B,N_i \rangle N_i(x)
\end{equation}
For many basis (orthonormal), basis and dual of the basis is same. Haar wavelet is one of the examples. 
Now consider a linear operator (integral) defined on $L^2$ . In particular we consider the radiosity
integral equation, which can be written as a linear operator operating on function $B(x)$\\
\begin{equation}
B(x)=E(x)+KB(x)
\end{equation}
where $KB(x) = \int K(x, y)B(y)dy$ is resulting function after operating on the function B(y). Since K (integration) is a linear operator it
is described by its action on our chosen basis. In other words, we
can write radiosity integral equation  kernel $K(x,y)$ as a matrix $\bf{K}$ of  infinite size. The coefficients of matrix $\bf{K}$  with
respect to our basis can be found by taking inner products with the dual functions. Thus the entries in the matrix representation
of $K(x,y)$ are given by\\
\begin{equation}\label{eq:kijcalc}
{\bf K}_{i,j}=\langle\, \langle K(x,y),N_j(y)\rangle,\tilde{N}_i(x) \,\rangle =\int \int K(x,y)N_j(y)\tilde{N}_i(x) dy\,dx
\end{equation}
with this matrix representation we write radiosity integral equation as a  matrix equation of infinite size\\
\begin{equation}\label{eq:systemeq}
\bf{B=E+KB}
\end{equation}
where, {\bf B} and {\bf E} are column matrix consisting of projection coefficients of $B(x)$ and $E(x)$ respectively, calculated by taking inner product with dual basis of space. Thus we have successfully discretized radiosity integral equation without any error in projection.

\section{Finite sized System of linear equation}
In previous section we have discretized the radiosity integral equation and we get alternative problem of solving system of linear equations. But we still can not solve it because it is problem is infinite size. To make the problem finite sized we select we select subspace of space in which problem is defined. To give an example, we can have subspace of function which are piecewise polynomial of order $m$. Basis for such space can be dilates and translates of function $\{1,x,x^2,...,x^m\}$ defined over every piece of domain of function space. In radiosity, we have finite domain of surfaces in scene, we can have basis space which is finite dimensional. consider radiosity function $B(x), \quad 0 \leq x < 1$. We can divide the domain into $n$ of size $\frac{1}{n}$ equal parts and assume that function is polynomial of upto order $m$ over this parts. One of the basis for this space is dilates and translates of function $\phi(x)=1, \quad \leq x < 1$. 
\begin{equation}\label{eq:scalediatrans}
    \{\phi_{i,n}(x)=\phi(nx-i)\},\quad i=0,1,...,n-1
\end{equation} 
Thus we have basis for piecewise polynomial (constant) function i.e. $m=1$. The function $\phi(x)=1, \quad \leq x < 1$ is actually Haar wavelet scaling function. If we increase value of m = 2, we have space of piecewise linear functions, and basis for this space is generated by dilates and translates (see Equation \ref{eq:scalediatrans}) of $\{1,x\}$. Dilates and translates of LLMW scaling functions $\phi^0(x)=1, \quad \leq x < 1$ and $\phi^1(x)=\sqrt{3}(2x-1), \quad \leq x < 1$ forms orthogonalized orthonormal basis function for same space.  If we select $m= 3$ we have space of piecewise quadratic functions and basis for it is dilates and translates of QLMW scaling functions (see Equations \ref{eq:qlmwphi0}, \ref{eq:qlmwphi0}, \ref{eq:qlmwphi0}). Note that once we select particular function space, there are multiple basis for that space. Method of projection requires orthogonal basis. 
\section{Algorithm}
In this section we describe steps of projection methods.
we first decide the finite dimension function space, a subset of function space of radiosity function ($L^2$). Since subspace is used for approximating the radiosity function, we select that function space which will have least error during  projection.  We then we select orthogonal basis for that space which will be used for getting projection coefficients $K_{i,j}$ and $E_{j}$ by taking inner product with basis as shown in Equation \ref{eq:kijcalc} \ref{eq:radproj}. Quadrature rules can be used to approximate inner products. Once we have the coefficients  we can solve system of linear equations shown in Equation \ref{eq:systemeq} using Jacobi iterative method. 


In next Chapters, Chapter \ref{ch:wavelets} and \ref{ch:waveletprojection} we introduce wavelets and the wavelet basis as an alternative basis to basis discussed in this chapter. Advantages of wavelet basis is also discussed in next chapters.
