

\chapter{\label{ch:projection}Projection Method for Radiosity Problem}
In chapter \ref{ch:problemformulatio} we saw that problem of radiosity was casted to integral equation but it is very difficult to solve the integral equation analytically due to geometry of scene. We can use projection methods to find approximate solution, Numerically. We first start with projecting the unknown radiosity function into finite dimension function space and integrate (operate kernel) on the projected radiosity function. After that we cast radiosity problem to system of linear equations which can be solved using computer. \\

\section{Projecting Radiosity Function}
We are working with linear operator (integral operator) defined on space of unknown function $B(x)$.  We first consider basis for this space. Since scene is finite so as the domain of surfaces, we are dealing with functions of finite energy, i.e. functions
in the linear vector space $L^2$ . There are many basis for
this space. Consider some general basis $\{N_i \}_{i \in \mathbb{N}}$ for $L^2$ space. Set contains infinitely many basis functions because $L^2$ is infinite dimensional space.
By definition, every function $B \in L^2$ can be written as a linear combination  $B(x) = \sum _i b_i N_i(x)$ of the basis functions for
some coefficients $b_i$ . In order to find these coefficients for projection of a given
function we define inner product on $L^2$ space. The inner product of
two functions F and G is defined as $\langle F, G \rangle  = \int F(x) G(y) dx$.
Given a function $B(x)$ and a set of basis functions function $N_i(x)$ , we can find the coefficients of $B(x)$ with respect to this basis is by
calculating inner products with the dual basis functions $\tilde{N}_i(x)$. Dual functions can be used as basis for the same space, and are defined
through the property
\begin{equation}
\langle N_i(x),\tilde{N}_j(x) \rangle =\delta_{ij}
\end{equation}
where $\delta_{ij}$ is the Kronecker Delta. Using the dual basis we can expand a function  $B(x)$ with respect to a original basis as
\begin{equation}
B(x) = \sum_ib_iN_i(x)= \sum_i\langle B,N_i \rangle N_i(x)
\end{equation}
For many basis, basis and dual of the basis is same. Haar wavelet is one of the examples. 
Now consider a linear operator (integral) defined on $L^2$ . In particular we consider the radiosity
integral equation, which can be written as a linear operator operating on function $B(x)$\\
\begin{equation}
B(x)=E(x)+KB(x)
\end{equation}
where $KB(x) = \int K(x, y)B(y)dy$ is resulting function after operating on the function B(y). Since K (integration) is a linear operator it
is described by its action on our chosen basis. In other words, we
can write radiosity integral equation  kernel $K(x,y)$ as a matrix $\bf{K}$ of  infinite size. The coefficients of matrix $\bf{K}$  with
respect to our basis can be found by taking inner products with the dual functions. Thus the entries in the matrix representation
of $K(x,y)$ are given by\\
\begin{equation}\label{eq:kijcalc}
{\bf K}_{i,j}=\langle\, \langle K(x,y),N_j(y)\rangle,\tilde{N}_i(x) \,\rangle =\int \int K(x,y)N_j(y)\tilde{N}_i(x) dy\,dx
\end{equation}
with this matrix representation we write radiosity integral equation as a  matrix equation of infinite size\\
\begin{equation}
\bf{B=E+KB}
\end{equation}
where, {\bf B} and {\bf E} are column matrix consisting of projection coefficients of $B(x)$ and $E(x)$ respectively, calulated by taking inner product with dual basis of space. 
\underline{ Show countably finite thing \\ and also show the difference between space and basis for that space}\\

in next section we introduce wavelets and the wavelet basis


\underline{tell about the desired space like piecewise polynmial and piecewise constant}
\\\\