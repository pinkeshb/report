%\section{Appendix}
\begin{appendix}
\appendixpage
\chapter{}
\pagestyle{empty}
\section{\label{apen:derivationdegenerate}Derivation of analytical solution of Fredholm IE with degenerate kernel}
\label{appendix-1}

%\subsection{\label{apn: Matrix F}Frequency Variance Measure for Orthogonal Design}
Equation \ref{Freq. Var. matrix elements} has been derived in this section. The DTFT of the $h(n) \in l^2(\mathbb{Z})$  

\begin{equation}
H(\omega)=\sum^{M}_{n=0}h(n) e^{-j\omega n}
\end{equation}
Recall,
\begin{equation*}
h(n)=0 ,\, \forall n \in \{ (n < 0) \cup (n > M)\}
\end{equation*}
The DTFT of $h(n)$ can be rewritten in matrix form as:
\begin{equation}
H(\omega)=\mathbf{h}^{T}\mathbf{e}(\omega)
\end{equation}

where,

\begin{equation}
\mathbf{e}(\omega)=[e^{-j\omega}\,\,\,\ e^{-j\omega 1}\,\,\,\ e^{-j\omega 2}\,\,\,\,\ldots\,\,\,\ e^{-j\omega M}]^T
\end{equation}

The frequency variance $\sigma_{\omega}^{2}$ is given by

\begin{eqnarray}
\sigma_{\omega}^{2}&=&\frac{1}{\pi} \int_{0}^{\pi}\omega^{2} |H(\omega)|d\omega\\
&=&\frac{1}{\pi} \int_{0}^{\pi}\omega^{2} \{\mathbf{h}^{T}\mathbf{e}(\omega)\mathbf{e}^{H}(\omega)\mathbf{h}\}d\omega\\
&=&\mathbf{h}^{T}\{\int_{0}^{\pi}\omega^{2}\mathbf{e}(\omega)\mathbf{e}^{H}(\omega)\frac{d\omega}{\pi}\}\mathbf{h}\\
&=&\mathbf{h}^{T}\{\int_{0}^{\pi}\omega^{2}\mathbf{E}(\omega)\frac{d\omega}{\pi}\}\mathbf{h}\\
&=&\mathbf{h}^{T}\mathbf{F}\mathbf{h}
\end{eqnarray}
where,
\begin{equation}
\mathbf{F}=\int_{0}^{\pi}\omega^{2}\mathbf{E}(\omega)\frac{d\omega}{\pi}
\label{eq:F equation}
\end{equation}
and
\begin{eqnarray}
\mathbf{E}(\omega)&=&
\begin{bmatrix}
    e^{-j\omega 0}
\\ e^{-j\omega 1} 
\\ \vdots
\\ e^{-j\omega M}   
\end{bmatrix}
\begin{bmatrix}
    e^{j\omega 0}
& e^{j\omega 1} 
& \ldots
& e^{j\omega M}   
\end{bmatrix}\\
&=&
\begin{bmatrix}
1 & e^{j\omega 1}& e^{j\omega 2} & \ldots & e^{j\omega M}\\
e^{-j\omega 1} &  1& e^{j\omega 1} & \ldots & e^{j\omega (M-1)}\\
e^{-j\omega 2} & e^{-j\omega 1}& 1 & \ldots & e^{j\omega (M-2)}\\
\vdots & \vdots& \vdots & \ddots & \vdots\\
e^{-j\omega M} & e^{-j\omega (M-1)}& e^{-j\omega (M-2)} & \ldots & 1\\
\label{eq:E matrix}
\end{bmatrix}
\end{eqnarray}
The element of the $\mathbf{E}$ matrix is $E_{k,l} =e^{j\omega (l-k)}$.
So when $k=l$ the element at that place is 1 which means all the diagonal elements are 1.

Now consider the integral
\begin{equation*}
\frac{1}{\pi}\int_{0}^{\pi}\omega^{2} e^{j\omega p}d\omega
\end{equation*}\\

For p=0,
\begin{equation}
\frac{1}{\pi}\int_{0}^{\pi}\omega^{2} e^{j\omega p}d\omega=\frac{1}{\pi}\int_{0}^{\pi}\omega^{2}d\omega=\frac{\omega^3}{3\pi}\biggr\vert_{0}^{\pi}=\frac{\pi^2}{3}.
\label{eq:p=0}
\end{equation}\\
For $p>0$,
\begin{eqnarray*}
\frac{1}{\pi} \int_{0}^{\pi}\omega^{2} e^{j\omega p}d\omega &=& \frac{\omega^{2} e^{j\omega p}}{jp}\biggr\vert_{0}^{\pi}-\int_{0}^{\pi}\frac{2\omega e^{j\omega p}}{jp}d\omega \\
&=& \frac{\omega^{2} e^{j\omega p}}{jp}\biggr\vert_{0}^{\pi}-\frac{2}{jp}\int_{0}^{\pi} \omega e^{j\omega p}d\omega \\
&=&\frac{\omega^{2} e^{j\omega p}}{jp}\biggr\vert_{0}^{\pi}-\frac{2}{jp}\biggr\{ \frac{\omega e^{j\omega p}}{jp}\biggr\vert_{0}^{\pi}- \int_{0}^{\pi} \frac{ e^{j\omega p}}{jp}d\omega \biggr\}\\
&=&\frac{\omega^{2} e^{j\omega p}}{jp}\biggr\vert_{0}^{\pi}-\frac{2}{(jp)^{2}}\biggr\{ \omega e^{j\omega p}- \frac{ e^{j\omega p}}{jp} \biggr\}\biggr\vert_{0}^{\pi}\\
&=&\frac{\omega^{2} e^{j\omega p}}{jp}\biggr\vert_{0}^{\pi}+\frac{2}{(p)^{2}}\biggr\{ \omega e^{j\omega p}- \frac{ e^{j\omega p}}{jp} \biggr\}\biggr\vert_{0}^{\pi}\\
&=& \frac{\pi^{2}(-1)^{p}}{jp}+\frac{2}{p^2}\biggr\{\pi e^{j\pi p} - \frac{e^{j \pi p}}{jp}\biggr\}-\biggr[0+\frac{2}{p^2}\biggr(\frac{-1}{jp}\biggr)\biggr]\\
\end{eqnarray*}

Hence for $p>0$,
\begin{equation}
\frac{1}{\pi}\int_{0}^{\pi}\omega^{2} e^{j\omega p}d\omega= \frac{\pi^{2}(-1)^{p}}{jp}+\frac{2\pi (-1)^{ p}}{p^2}-\frac{2(-1)^p}{jp^3}+\frac{2}{jp^3}
\label{eq:p>0}
\end{equation}
From expressions \ref{eq:F equation}, \ref{eq:E matrix},\ref{eq:p=0} and \ref{eq:p>0}, we can evaluate the $\mathbf{F}$ matrix. The expression for the $\mathbf{F}$ matrix is given below.

 \begin{eqnarray}
 \label{Freq. Var. matrix}
 \mathbf{F_{k,l}} =  \begin{cases}
 \frac{\pi^2}{3}\,\,\, \text{if}\,\, k=l \\ 
 \frac{\pi^2{(-1)^{(l-k)}}}{j(l-k)} + \frac{2\pi{(-1)^{(l-k)}}}{(l-k)^2} + \frac{2\left(1 - {(-1)^{(l-k)}}\right)}{j{(l-k)^3}}\,\,\, \text{if}\,\, l \neq k
 \end{cases}
 \end{eqnarray}
\end{appendix}