\chapter{\label{Intro}Introduction}

The rendering problem of calculating the radiance  (brightness) of all surfaces in 3D scene is very important in 3D computer graphics. Given a geometrical location and optical characteristics  (reflectivity) of surfaces along with light sources (emitters) in 3D scene, our concern is to calculate radiance and generate realistic looking image of scene, from particular viewpoint, illuminated with light source.

\section{Direct and Indirect illumination}
A renderer,in computer graphics, is algorithm which generates images from of scene. There are two types of renderer, direct illumination renderer and indirect illumination renderer. 
Figure \ref{fig:directindirect} shows an example of difference between direct and indirect illumination. The image on the left in the figure is rendered with standard direct illumination renderer. Illumination of surface is done by two things, first one is light received directly from light source and second one is ambient light without which surfaces which are not directly illuminated will be completely dark. Image on the right of Figure \ref{fig:directindirect} is rendered by indirect illumination renderer. Clearly, image generate using indirect illumination renderer are more realistic.
\begin{figure}[tbh]
\centering{}
\captionsetup{justification=centering}
\includegraphics[width=5in]{DirectIndirect.jpg}
\caption{\label{fig:directindirect}Difference between direct illumination and indirect illumination \cite{wikiradiosity}}
\end{figure}

Global illumination or indirect illumination is a name for a group of algorithms  (renderer) used in 3D computer graphics that are meant to add more realistic lighting to 3D scenes. Such algorithms take into account not only the light energy which comes directly from a light source  (direct illumination), but also light energy from the same source which are reflected by other surfaces in the scene  (indirect illumination). As a result, global illumination adds realistic features like light bouncing  (multiple inter-reflection between two surfaces) and color bleeding between pair of adjacent surfaces. A well known global illumination algorithm is ray-tracing \cite{Whitted}, which computes  radiance of a area of a surface of a scene seen from one given viewpoint. In other words, it only calculates radiance of points which are visible in final image. Thus for changing the viewpoint of scene we need to run algorithm again, making the algorithm viewpoint dependent.

\section{Radiosity}

Radiosity algorithms are set of global illumination algorithms which calculates radiance of surfaces reflecting light diffusely (emitting equal apparent brightness in all the directions above the surface). In other words, radiosity takes the 3D scenes with Lambertian surfaces as input. Lambertian reflectance is the property of an ideal "matte" or diffusely reflecting surface. Opposite to ray-tracing, radiosity calculates radiance of all points in the scene. This allows the computation to be viewpoint independent i.e. solution for given scene can be used to generate images of scene from arbitrary viewpoint. This is advantage as well as drawback as we need to do extra work to find radiance of all the points. Radiosity also refers to measure of power per unit area at a point. Thus solution of the this problem is radiosity function over the domain of surfaces in the scene. 




\section{Organization of the Report}
We first discuss literature survey and approaches taken to solve the problem in Chapter \ref{ch:literature}. The we formally define radiosity problem and associated radioity integral equation, in Chapter \ref{ch:problemformulation}. Then projection method, a numerical solution, is discussed in Chapters \ref{ch:projection}, \ref{ch:wavelets} and \ref{ch:waveletprojection} with different basis discussed in Chapter \ref{ch:wavelets}. Finally we discuss results and comparison of wavelet in Chapter \ref{ch:experimentandresult}.